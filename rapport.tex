\documentclass[a4paper]{article} 
\addtolength{\hoffset}{-2.25cm}
\addtolength{\textwidth}{4.5cm}
\addtolength{\voffset}{-3.25cm}
\addtolength{\textheight}{5cm}
\setlength{\parskip}{0pt}
\setlength{\parindent}{0in}

%----------------------------------------------------------------------------------------
%	PACKAGES AND OTHER DOCUMENT CONFIGURATIONS
%----------------------------------------------------------------------------------------

\usepackage{charter} % Use the Charter font
\usepackage[utf8]{inputenc} % Use UTF-8 encoding
\usepackage{microtype} % Slightly tweak font spacing for aesthetics
\usepackage[english,french]{babel} % Language hyphenation and typographical rules
\usepackage{amsthm, amsmath, amssymb} % Mathematical typesetting
\usepackage{float} % Improved interface for floating objects
\usepackage[final, colorlinks = true, 
            linkcolor = blue, 
            citecolor = lightblue]{hyperref} % For hyperlinks in the PDF
\usepackage{graphicx, multicol} % Enhanced support for graphics
\usepackage{xcolor} % Driver-independent color extensions
\usepackage{listings, style/lstlisting} % Environment for non-formatted code, !uses style file!
\usepackage{pseudocode} % Environment for specifying algorithms in a natural way
\usepackage[backend=biber,style=numeric,
            sorting=nyt]{biblatex} % Complete reimplementation of bibliographic facilities
\addbibresource{ecl.bib}
\usepackage{csquotes} % Context sensitive quotation facilities
\usepackage{fancyhdr} % Headers and footers
\pagestyle{fancy} % All pages have headers and footers
\fancyhead{}\renewcommand{\headrulewidth}{0pt} % Blank out the default header
\fancyfoot[L]{} % Custom footer text
\fancyfoot[C]{} % Custom footer text
\fancyfoot[R]{\thepage} % Custom footer text
\newcommand{\note}[1]{\marginpar{\scriptsize \textcolor{red}{#1}}} % Enables comments in red on margin
%----------------------------------------------------------------------------------------

\begin{document}

%-------------------------------
%	TITLE SECTION
%-------------------------------

\fancyhead[C]{}
\hrule \medskip % Upper rule
\begin{minipage}{0.295\textwidth} 
\raggedright
\footnotesize
Eric Benhamou, \hfill\\   
Marie Guerin, \hfill\\   
Juliette Pansart, \hfill\\
Valentin Melot  \hfill\\
\end{minipage}
\begin{minipage}{0.4\textwidth} 
\centering 
\large 
Application to Journalism\\ 
\normalsize 
M2 MASH \\ 
Robin Ryder
\end{minipage}
\begin{minipage}{0.295\textwidth} 
\raggedleft
\today\hfill\\
\end{minipage}
\medskip\hrule 
\bigskip

%-------------------------------
%	CONTENTS
%-------------------------------

\section{Context}
The objective of this orignanl and fruitful course was to set up an interaction between mathematics and journalists students in collaboration with the Institut Pratique du Journalisme. After bing presented the problematic of Safety on roads, with a key focuss on departemental road, we formed a group of 6 people (the 4 of us and 2 students from M2 IPJ: Cloe Ant and Antoine Cadaux) to analyze data from the \href{https://www.data.gouv.fr/fr/datasets/base-de-donnees-accidents-corporels-de-la-circulation/}{ONISIR data set}. We were asked to collaborate to find a journalistic problematic and illustrate the point of view with our statistical skills, in a traditional data journalism course. The overall idea was to study the data, find a problematic, propose and validate relevant models, perform mathematical analysis, choose an angle, develop some data visualizations. The journalist students were asked to write a report accessible to the general public in the form of a press article, while we were asked to write a more scientific one exposing our approach. We had 3 meeting together at the IPJ (Jan 16th, Feb 6th, and 27th). We summarize below our approach and findings and took the initiative to illustrate our point of view with relevant python code section and graphics. Full code is available at \href{https://github.com/ericbenhamou/MASH_IPJ_2018}{githubt}

\subsection{Subject}
We offer to study the mortality on road with a particular focuss on departements (which was the keyword drawn randomly for our group). The theme is very up to date with the recent announcement of the prime minister Edouard Philippe abou thte speed reduction from 90 to 80 km/h on two way national and departemental roads (catr 2 and 3) in the database, which are equiped wit a central seperator (circ other than 3).

The key findings are that these roads are actually the most dangerous according to the database of the ONISR (when looking at the criterium of danger: if you have an accident, proportion of serious or dead injuries). We try to give some hint about the impact of the measre (should it reduce mortality as much as promised). We also spent time to look at the statistics on these roads get an intuition of the causes and sources of accidents as well as possible solutions

\subsection{Description of the dataset}
The dataset about personal injury of traffic in France is split into 4 diferent datasets listed by years on the website of the public adminstration and refered to as the  \href{https://www.data.gouv.fr/fr/datasets/base-de-donnees-accidents-corporels-de-la-circulation/}{ONISIR data set} given in csv format. In addition, a pdf file is given to describe the various fields and codes used in these datasets. We investigated the data of year \textbf{2016}. Let us give briefly the caracteristics:
\\
\\
The four sub databases are with self explained names
\begin{itemize}
\item vehicules 
\item usagers
\item lieux
\item caracteristiques
\end{itemize}

They are easily loaded with pandas python library


\bigskip

%------------------------------------------------

\end{document}
